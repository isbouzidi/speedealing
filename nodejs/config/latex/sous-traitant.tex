\documentclass[a4paper, oneside, 12pt, french]{article}

\usepackage[a4paper]{geometry}
\geometry{top=2cm, bottom=1cm, left=1.5cm, right=1.5cm,foot=1.5cm,head=2cm,headsep=.5cm,includefoot}

\usepackage{symeos_article}

\newcommand{\CLIENT}{}
\newcommand{\TYPE}{Fiche prospect}

\newcommand{\NUMREV}{Version 1.0}
\newcommand{\DATEREV}{\today}

\begin{document}

\begin{center}
\huge \bf \TYPE \\
\textcolor{Symeos_Red}{$\bullet \bullet$}\textcolor{Symeos_Grey}{$\bullet$}\\
\small \textit{\NUMREV \ - \DATEREV} 
\end{center}

\begin{fminipage}
\begin{minipage}{.5\textwidth}
Auteur : 
\newline
\end{minipage}
\hspace{2mm}
\begin{minipage}{0.48\textwidth}
Date :
\newline
\end{minipage}
\end{fminipage}

{\Large Les coordonnées du prospect}

\begin{fminipage}
\vspace{12pt}
Entreprise : \\
\hspace*{10cm} Site Internet : \\
Adresse : \\
\hspace*{10cm} Tél : \\
\vspace{36pt}
\end{fminipage}

{\Large La présentation rapide du prospect}

\begin{fminipage}
\vspace{12pt}
Activité : 
\newline
Nb utilisateurs : 
\newline
Environnement économique du prospect : \textit{(Positionnement, points faibles/fort)}
\newline
\newline
Interlocuteurs : \textit{noms, coordonnées, fonction, pouvoir de décision, centres d'intérêts}
\newline
\newline
\newline
\newline
Origine du contact :
\end{fminipage}

{\Large Les opportunités envisageables d'affaires}

\begin{fminipage}
\vspace{12pt}
Intérêt global que représente ce propect pour Symeos : \textit{aborder un nouveau secteur, remplacement produit,\dots}
\newline
\newline
Possibiliés  : commerciales envisageables : \textit{volume de commandes\dots}
\newline
\newline
Contraintes probables pour Symeos : \textit{prix, livraison, applicances,\dots}
\newline
\newline
Fournisseurs et/ou produits (volumes) actuels :\\
\end{fminipage}

\newpage

{\Large L'historique des contacts}

\small
\begin{tabular}{|c|p{3cm}|p{3cm}|p{5cm}|p{3cm}|}
\hline
Date & Interlocuteur(s) Symeos & Interlocuteur(s) chez le prospect& Résumé du contact & Suite programmée : \textit{envoi doc, RDV,relances\dots}\\
\hline
\hline
& & & & \\
& & & & \\
& & & & \\
\hline
& & & & \\
& & & & \\
& & & & \\
\hline
& & & & \\
& & & & \\
& & & & \\
\hline
& & & & \\
& & & & \\
& & & & \\
\hline
& & & & \\
& & & & \\
& & & & \\
\hline
& & & & \\
& & & & \\
& & & & \\
\hline
& & & & \\
& & & & \\
& & & & \\
\hline
& & & & \\
& & & & \\
& & & & \\
\hline
& & & & \\
& & & & \\
& & & & \\
\hline
& & & & \\
& & & & \\
& & & & \\
\hline
& & & & \\
& & & & \\
& & & & \\
\hline
& & & & \\
& & & & \\
& & & & \\
\hline
& & & & \\
& & & & \\
& & & & \\
\hline
& & & & \\
& & & & \\
& & & & \\
\hline
\end{tabular}

\newpage

\begin{center}
\huge \bf Compte rendu d'une action de prospection \\
\textcolor{Symeos_Red}{$\bullet \bullet$}\textcolor{Symeos_Grey}{$\bullet$}\\
\small \textit{\NUMREV \ - \DATEREV} 
\end{center}

\normalsize

{\Large Le contexte de l'action}

\begin{fminipage}
\vspace{12pt}
Type : (\textit{Appel téléphonique, réunion, télé-conférence,\dots}\\
\newline
Date : \\
\hspace*{10cm} Lieu : \\
Interlocuteurs : \\
\newline
Origine de l'action :\\
\newline
\end{fminipage}

{\Large Les informations à échanger}

\begin{fminipage}
\vspace{12pt}
Objectif de l'action pour Symeos : \\
\newline
\newline
Objectif de l'action pour le client :\\
\newline
\newline
Les informations que nous devons absolument obtenir :\\
\newline
\newline
Les informations spécifiques à délivrer au prospect : \\
\newline
\newline
Les objections du client :\\
\newline
\newline
La réponse aux objections :\\
\newline
\newline
Moyens pour satisfaire le client :\\
\newline
\newline
\end{fminipage}

%{\Large Notre marge de man\oe uvre en terme d'offre commerciale}

%\begin{fminipage}
%\vspace{12pt}
%Le contenu global de l'offre que nous allons proposer :\\
%\newline
%Les éléments d'offre éventuellement négociable (\textit{installation gratuite, formation,\dots}) : \\
%\newline
%Les éléments d'offre non négociables (textit{délai de règlement,\dots})
%\newline
%\end{fminipage}

%{\Large Notre marge de man\oe uvre en terme de prix}

%\begin{fminipage}
%\vspace{12pt}
%Le prix idéal auquel nous souhaitons traiter l'affaire :\\
%\newline
%Le prix acceptable pour Symeos : \\
%\newline
%Le prix plancher en dessous duquel nous ne donnerons pas suite :
%\newline
%\end{fminipage}

\newpage

\begin{center}
\huge \bf Guide de prospection \\
\textcolor{Symeos_Red}{$\bullet \bullet$}\textcolor{Symeos_Grey}{$\bullet$}\\
\small \textit{\NUMREV \ - \DATEREV} 
\end{center}

\begin{fminipage}
\begin{itemize}
    \item les questions relatives au contexte : elles portent sur l'environnement du client visé et de l'entreprise qu'il représente :
    \begin{itemize}
          \item êtes-vous déjà client de ce type de produit/service ?
          \item quel type d'équipement possédez-vous ?
          \item quelles personnes décident avec vous ?
          \item quelle est la taille de\dots (entreprise, chiffre d'affaire, volumes users, \dots)
    \end{itemize}
    \item les questions sur la problématique connue : elles concernent les besoins à satisfaire, les problèmes rencontrés, le but recherché\dots :
    \begin{itemize}
          \item dans quelle situation utilisez-vous ce produit/service ?
          \item de quelle façon remplit il son rôle ?
          \item qu'est-ce qui est satisfaisant/pas satisfaisant dans la situation actuelle ?
          \item atteignez-vous facilement vos objectifs ?
          \item que recherchez-vous ?
    \end{itemize}
    \item les questions sur les enjeux : il ne suffit pas de détecter un besoin, encore faut il que votre interlocuteur ai envie de la satisfaire :
    \begin{itemize}
          \item dans ce que nous venons d'aborder ensemble qu'est-ce qui est important à vos yeux ?
          \item quelles sont les conséquence si ce problème n'est pas réglé (c'est le coût de la non qualité) ?
          \item si vous ne prenez pas telle mesure alors que ferez-vous ?
    \end{itemize}
    \item les questions sur les solutions : elles amènent le client à se projeter dans l'avenir et de sonder ses idées :
    \begin{itemize}
          \item quelles solutions avez-vous envisagées ?
          \item pour vous quels seraient l'intérêt et les limites de ces solutions ?
          \item dans l'idéal, sans qu'il ne soit question de budget ou de faisabilité, quelle serait la solution idéale (si vous aviez une baguette magique que changeriez-vous) ?
    \end{itemize}
\end{itemize}

\end{fminipage}



\end{document}

